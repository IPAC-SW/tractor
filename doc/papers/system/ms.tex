%%%%%%%%%%%%%%%%%%%%%%%%%%%%%%%%%%%%%%%%%%%%%%%%%%%%%%%%%%%%%%%%%%%%%%%%%%%%%%
% 
% The Tractor Paper
% 
%%%%%%%%%%%%%%%%%%%%%%%%%%%%%%%%%%%%%%%%%%%%%%%%%%%%%%%%%%%%%%%%%%%%%%%%%%%%%%

\documentclass[useAMS,usenatbib]{mn2e}

\voffset=-0.8in

% Packages:
\usepackage{graphicx}
\usepackage{amsmath}
\usepackage{xspace}

% Macros:
% JOURNALS
\newcommand {\apj} {ApJ}
\newcommand {\apjl} {ApJL}
\newcommand {\apjs} {ApJS}
\newcommand {\mnras} {MNRAS}
\newcommand {\apss} {Ap \& SS}
\newcommand {\aap} {A\&A}
\newcommand {\aj} {AJ}
\newcommand {\prd} {Phys. Rev. D}
\newcommand {\nat} {Nature}
\newcommand {\araa} {ARA\&A}
\newcommand {\jgr} {J. Geophys. Res.}
\newcommand {\pasp} {PASP}

% MISC
\newcommand {\etal} {et~al.~}
\newcommand {\lta} {\mathrel{\spose{\lower 3pt\hbox{$\sim$}}\raise  2.0pt\hbox{$<$}}}
\newcommand {\gta} {\mathrel{\spose{\lower  3pt\hbox{$\sim$}}\raise 2.0pt\hbox{$>$}}}
\def\Sref#1{Section~\ref{#1}\xspace}
\def\Fref#1{Figure~\ref{#1}\xspace}
\def\Tref#1{Table~\ref{#1}\xspace}
\def\Eref#1{Equation~\ref{#1}\xspace}

% UNITS
\newcommand {\kms} {\ifmmode  \,\rm km\,s^{-1} \else $\,\rm km\,s^{-1}  $ \fi }
\newcommand {\kpc} {\ifmmode  {\rm kpc}  \else ${\rm  kpc}$ \fi  }  
\newcommand {\pc} {\ifmmode  {\rm pc}  \else ${\rm pc}$ \fi  }  
\newcommand {\Msun} {\ifmmode {\rm M_{\odot}} \else ${\rm M_{\odot}}$ \fi} 
\newcommand {\Zsun} {\ifmmode {\rm Z_{\odot}} \else ${\rm Z_{\odot}}$ \fi} 
\newcommand {\yr} {\ifmmode yr^{-1} \else $yr^{-1}$ \fi} 
\newcommand {\hMsun} {\ifmmode h^{-1}\,\rm M_{\odot} \else $h^{-1}\,\rm M_{\odot}$ \fi}

% NOTATION

% SOFTWARE/HARDWARE
\def\SExtractor{{\sc SExtractor}\xspace}
\def\hst{{\it HST}\xspace}
\def\galfit{{\sc galfit}\xspace}
\def\python{{\sc python}\xspace}

% PROBABILITY THEORY
\def\pr{{\rm Pr}}
\def\data{{\mathbf{d}}}
\def\datap{{\mathbf{d}^{\rm p}}}
\def\datai{d_i}
\def\datapi{d^{\rm p}_i}
\def\pars{\boldsymbol{\theta}}


% COMMENTING
\usepackage[usenames]{color}
\newcommand{\comment}[2]{\textcolor{blue}{\bf #1: #2}}
\newcommand{\problem}[2]{\textcolor{red}{\bf #1: #2}}
\newcommand{\todo}[2]{{\bf TO-DO: #1: #2}}

\def\firstplace{Institute 1, University 1, City 1, Country 1}
\def\secondplace{Institute 2, University 2, City 2, Country 2}
\def\thirdplace{Institute 3, University 3, City 3, Country 3}
\def\fourthplace{Institute 4, University 4, City 4, Country 4}
\def\fifthplace{Institute 5, University 5, City 5, Country 5}
\def\sixthplace{Institute 6, University 6, City 6, Country 6}

\def\email{\tt first.author@domain.name}


%%%%%%%%%%%%%%%%%%%%%%%%%%%%%%%%%%%%%%%%%%%%%%%%%%%%%%%%%%%%%%%%%%%%%%%%%%%%%%

\title[The Tractor]
{The Tractor: a code for probabilistic modeling of astronomical images}
    
\author[All of us]{%
  Dustin Lang,$^{1}$\thanks{\email}
  David W. Hogg,$^{2}$
  Third Author,$^{3}$
\newauthor{%
  Fourth Author,$^{4}$
  Fifth Author,$^{5}$
  Sixth Author.$^{6}$}
  \medskip\\
  $^1$\cmu\\
  $^2$\nyu\\
  $^3$\thirdplace\\
  $^4$\fourthplace\\
  $^5$\fifthplace\\
  $^6$\sixthplace\\
}

%%%%%%%%%%%%%%%%%%%%%%%%%%%%%%%%%%%%%%%%%%%%%%%%%%%%%%%%%%%%%%%%%%%%%%%%%%%%%%

\begin{document}
             
\date{To be submitted to MNRAS}
             
\pagerange{\pageref{firstpage}--\pageref{lastpage}}\pubyear{2XXX}

\maketitle

\label{firstpage}

%%%%%%%%%%%%%%%%%%%%%%%%%%%%%%%%%%%%%%%%%%%%%%%%%%%%%%%%%%%%%%%%%%%%%%%%%%%%%%

\begin{abstract}

We present a new open-source software package, The Tractor, which
enables probabilistic modeling of astronomical images.  The approach
is \emph{generative}: the aim is to produce a pixel-space models of
the observed images given parameters that describe the astronomical
sources and the calibration properties of the images.  We typically
assume pixelwise independent Gaussian noise, so the log-likelihood is
a chi-squared distribution.
%
General S\'ersic galaxy models are supported.  We make heavy use of
mixture-of-Gaussian approximations, relying on the analytic
convolution of Gaussian representations of the point-spread function
and galaxy profile to enable rapid pixel-by-pixel rendering of the
model images.
%
%  We use a mixture of Gaussians
%to describe both galaxies and point spread functions, and compare these models
%directly to the pixels of all available images via the joint likelihood
%function.
%
The code includes a simple optimiser for locating the peak of the
posterior probability for the model parameters, and can also call an
MCMC sampler for characterising the full distribution.  This allows,
among other things, calibration and other nuisance parameters to be
marginalized out.  We demonstrate the Tractor's performace in several
different settings, including galaxy shape and photometry inference,
in heterogeneous datasets involving multiple bandpasses, observing
epochs, instrumentation, and observing conditions. We show that...

% We don't go into much detail, simply referring to MBI1.  I think we have lots of other
% content, so there's no need to talk about our Great3 efforts in any depth.
\end{abstract}

\begin{keywords}
methods: data analysis --- surveys --- techniques: image processing
\end{keywords}

%%%%%%%%%%%%%%%%%%%%%%%%%%%%%%%%%%%%%%%%%%%%%%%%%%%%%%%%%%%%%%%%%%%%%%%%%%%%%%

\section{Introduction}

Goes here.

We are interested in the following questions:

\begin{itemize}

\item First question?

\item Second question? 

\item Third question? 

\end{itemize}

This paper is organized as follows. In \Sref{sec:first} we \ldots
In \Sref{sec:discuss} we present our conclusions.


%%%%%%%%%%%%%%%%%%%%%%%%%%%%%%%%%%%%%%%%%%%%%%%%%%%%%%%%%%%%%%%%%%%%%%%%%%%%%%

\section{Next section}
\label{sec:next}


%%%%%%%%%%%%%%%%%%%%%%%%%%%%%%%%%%%%%%%%%%%%%%%%%%%%%%%%%%%%%%%%%%%%%%%%%%%%%%

\section{Conclusions}
\label{sec:conclude}

Summarize briefly.

Our conclusions regarding \ldots can be stated as follows:

\begin{itemize}

\item First answer.

\item Second answer. 

\item Third answer. 

\end{itemize}

Wrap up.


%%%%%%%%%%%%%%%%%%%%%%%%%%%%%%%%%%%%%%%%%%%%%%%%%%%%%%%%%%%%%%%%%%%%%%%%%%%%%%

\section*{Acknowledgements}

We thank A, B and C for useful discussions.
%
XXX acknowledges support from XXX.
%
YYY acknowledges support from YYY.



%%%%%%%%%%%%%%%%%%%%%%%%%%%%%%%%%%%%%%%%%%%%%%%%%%%%%%%%%%%%%%%%%%%%%%%%%%%%%%
% MNRAS does not use bibtex, input .bbl file instead. Generate this in the
% makefile using bubble script in scriptutils:

%%%%%%%%%%%%%%%%%%%%%%%%%%%%%%%%%%%%%%%%%%%%%%%%%%%%%%%%%%%%%%%%%%%%%%%%%%%%%%
% 
% GREAT3 Weak Lensing Challenge Entry
% 
%%%%%%%%%%%%%%%%%%%%%%%%%%%%%%%%%%%%%%%%%%%%%%%%%%%%%%%%%%%%%%%%%%%%%%%%%%%%%%

\documentclass[useAMS,usenatbib]{mn2e}

\voffset=-0.8in

% Packages:
\usepackage{graphicx}
\usepackage{amsmath}
\usepackage{xspace}

% Macros:
% JOURNALS
\newcommand {\apj} {ApJ}
\newcommand {\apjl} {ApJL}
\newcommand {\apjs} {ApJS}
\newcommand {\mnras} {MNRAS}
\newcommand {\apss} {Ap \& SS}
\newcommand {\aap} {A\&A}
\newcommand {\aj} {AJ}
\newcommand {\prd} {Phys. Rev. D}
\newcommand {\nat} {Nature}
\newcommand {\araa} {ARA\&A}
\newcommand {\jgr} {J. Geophys. Res.}
\newcommand {\pasp} {PASP}

% MISC
\newcommand {\etal} {et~al.~}
\newcommand {\lta} {\mathrel{\spose{\lower 3pt\hbox{$\sim$}}\raise  2.0pt\hbox{$<$}}}
\newcommand {\gta} {\mathrel{\spose{\lower  3pt\hbox{$\sim$}}\raise 2.0pt\hbox{$>$}}}
\def\Sref#1{Section~\ref{#1}\xspace}
\def\Fref#1{Figure~\ref{#1}\xspace}
\def\Tref#1{Table~\ref{#1}\xspace}
\def\Eref#1{Equation~\ref{#1}\xspace}

% UNITS
\newcommand {\kms} {\ifmmode  \,\rm km\,s^{-1} \else $\,\rm km\,s^{-1}  $ \fi }
\newcommand {\kpc} {\ifmmode  {\rm kpc}  \else ${\rm  kpc}$ \fi  }  
\newcommand {\pc} {\ifmmode  {\rm pc}  \else ${\rm pc}$ \fi  }  
\newcommand {\Msun} {\ifmmode {\rm M_{\odot}} \else ${\rm M_{\odot}}$ \fi} 
\newcommand {\Zsun} {\ifmmode {\rm Z_{\odot}} \else ${\rm Z_{\odot}}$ \fi} 
\newcommand {\yr} {\ifmmode yr^{-1} \else $yr^{-1}$ \fi} 
\newcommand {\hMsun} {\ifmmode h^{-1}\,\rm M_{\odot} \else $h^{-1}\,\rm M_{\odot}$ \fi}

% NOTATION

% SOFTWARE/HARDWARE
\def\SExtractor{{\sc SExtractor}\xspace}
\def\hst{{\it HST}\xspace}
\def\galfit{{\sc galfit}\xspace}
\def\python{{\sc python}\xspace}

% PROBABILITY THEORY
\def\pr{{\rm Pr}}
\def\data{{\mathbf{d}}}
\def\datap{{\mathbf{d}^{\rm p}}}
\def\datai{d_i}
\def\datapi{d^{\rm p}_i}
\def\pars{\boldsymbol{\theta}}


% COMMENTING
\usepackage[usenames]{color}
\newcommand{\comment}[2]{\textcolor{blue}{\bf #1: #2}}
\newcommand{\problem}[2]{\textcolor{red}{\bf #1: #2}}
\newcommand{\todo}[2]{{\bf TO-DO: #1: #2}}

\def\firstplace{Institute 1, University 1, City 1, Country 1}
\def\secondplace{Institute 2, University 2, City 2, Country 2}
\def\thirdplace{Institute 3, University 3, City 3, Country 3}
\def\fourthplace{Institute 4, University 4, City 4, Country 4}
\def\fifthplace{Institute 5, University 5, City 5, Country 5}
\def\sixthplace{Institute 6, University 6, City 6, Country 6}

\def\email{\tt first.author@domain.name}


%%%%%%%%%%%%%%%%%%%%%%%%%%%%%%%%%%%%%%%%%%%%%%%%%%%%%%%%%%%%%%%%%%%%%%%%%%%%%%

\title[Short title]
{The long version of the title of this paper}
    
\author[All of us]{%
  First Author,$^{1}$\thanks{\email}
  Second Author,$^{2}$
  Third Author,$^{3}$
\newauthor{%
  Fourth Author,$^{4}$
  Fifth Author,$^{5}$
  Sixth Author.$^{6}$}
  \medskip\\
  $^1$\firstplace\\
  $^2$\secondplace\\
  $^3$\thirdplace\\
  $^4$\fourthplace\\
  $^5$\fifthplace\\
  $^6$\sixthplace\\
}

%%%%%%%%%%%%%%%%%%%%%%%%%%%%%%%%%%%%%%%%%%%%%%%%%%%%%%%%%%%%%%%%%%%%%%%%%%%%%%

\begin{document}
             
\date{To be submitted to MNRAS}
             
\pagerange{\pageref{firstpage}--\pageref{lastpage}}\pubyear{2XXX}

\maketitle

\label{firstpage}

%%%%%%%%%%%%%%%%%%%%%%%%%%%%%%%%%%%%%%%%%%%%%%%%%%%%%%%%%%%%%%%%%%%%%%%%%%%%%%

\begin{abstract}
Write this first.
\end{abstract}

\begin{keywords}
  Need keywords.
\end{keywords}

%%%%%%%%%%%%%%%%%%%%%%%%%%%%%%%%%%%%%%%%%%%%%%%%%%%%%%%%%%%%%%%%%%%%%%%%%%%%%%

\section{Introduction}

Goes here.

We are interested in the following questions:

\begin{itemize}

\item First question?

\item Second question? 

\item Third question? 

\end{itemize}

This paper is organized as follows. In \Sref{sec:first} we \ldots
In \Sref{sec:discuss} we present our conclusions.


%%%%%%%%%%%%%%%%%%%%%%%%%%%%%%%%%%%%%%%%%%%%%%%%%%%%%%%%%%%%%%%%%%%%%%%%%%%%%%

\section{Next section}
\label{sec:next}


%%%%%%%%%%%%%%%%%%%%%%%%%%%%%%%%%%%%%%%%%%%%%%%%%%%%%%%%%%%%%%%%%%%%%%%%%%%%%%

\section{Conclusions}
\label{sec:conclude}

Summarize briefly.

Our conclusions regarding \ldots can be stated as follows:

\begin{itemize}

\item First answer.

\item Second answer. 

\item Third answer. 

\end{itemize}

Wrap up.


%%%%%%%%%%%%%%%%%%%%%%%%%%%%%%%%%%%%%%%%%%%%%%%%%%%%%%%%%%%%%%%%%%%%%%%%%%%%%%

\section*{Acknowledgements}

We thank A, B and C for useful discussions.
%
XXX acknowledges support from XXX.
%
YYY acknowledges support from YYY.



%%%%%%%%%%%%%%%%%%%%%%%%%%%%%%%%%%%%%%%%%%%%%%%%%%%%%%%%%%%%%%%%%%%%%%%%%%%%%%
% MNRAS does not use bibtex, input .bbl file instead. Generate this in the
% makefile using bubble script in scriptutils:

%%%%%%%%%%%%%%%%%%%%%%%%%%%%%%%%%%%%%%%%%%%%%%%%%%%%%%%%%%%%%%%%%%%%%%%%%%%%%%
% 
% GREAT3 Weak Lensing Challenge Entry
% 
%%%%%%%%%%%%%%%%%%%%%%%%%%%%%%%%%%%%%%%%%%%%%%%%%%%%%%%%%%%%%%%%%%%%%%%%%%%%%%

\documentclass[useAMS,usenatbib]{mn2e}

\voffset=-0.8in

% Packages:
\usepackage{graphicx}
\usepackage{amsmath}
\usepackage{xspace}

% Macros:
% JOURNALS
\newcommand {\apj} {ApJ}
\newcommand {\apjl} {ApJL}
\newcommand {\apjs} {ApJS}
\newcommand {\mnras} {MNRAS}
\newcommand {\apss} {Ap \& SS}
\newcommand {\aap} {A\&A}
\newcommand {\aj} {AJ}
\newcommand {\prd} {Phys. Rev. D}
\newcommand {\nat} {Nature}
\newcommand {\araa} {ARA\&A}
\newcommand {\jgr} {J. Geophys. Res.}
\newcommand {\pasp} {PASP}

% MISC
\newcommand {\etal} {et~al.~}
\newcommand {\lta} {\mathrel{\spose{\lower 3pt\hbox{$\sim$}}\raise  2.0pt\hbox{$<$}}}
\newcommand {\gta} {\mathrel{\spose{\lower  3pt\hbox{$\sim$}}\raise 2.0pt\hbox{$>$}}}
\def\Sref#1{Section~\ref{#1}\xspace}
\def\Fref#1{Figure~\ref{#1}\xspace}
\def\Tref#1{Table~\ref{#1}\xspace}
\def\Eref#1{Equation~\ref{#1}\xspace}

% UNITS
\newcommand {\kms} {\ifmmode  \,\rm km\,s^{-1} \else $\,\rm km\,s^{-1}  $ \fi }
\newcommand {\kpc} {\ifmmode  {\rm kpc}  \else ${\rm  kpc}$ \fi  }  
\newcommand {\pc} {\ifmmode  {\rm pc}  \else ${\rm pc}$ \fi  }  
\newcommand {\Msun} {\ifmmode {\rm M_{\odot}} \else ${\rm M_{\odot}}$ \fi} 
\newcommand {\Zsun} {\ifmmode {\rm Z_{\odot}} \else ${\rm Z_{\odot}}$ \fi} 
\newcommand {\yr} {\ifmmode yr^{-1} \else $yr^{-1}$ \fi} 
\newcommand {\hMsun} {\ifmmode h^{-1}\,\rm M_{\odot} \else $h^{-1}\,\rm M_{\odot}$ \fi}

% NOTATION

% SOFTWARE/HARDWARE
\def\SExtractor{{\sc SExtractor}\xspace}
\def\hst{{\it HST}\xspace}
\def\galfit{{\sc galfit}\xspace}
\def\python{{\sc python}\xspace}

% PROBABILITY THEORY
\def\pr{{\rm Pr}}
\def\data{{\mathbf{d}}}
\def\datap{{\mathbf{d}^{\rm p}}}
\def\datai{d_i}
\def\datapi{d^{\rm p}_i}
\def\pars{\boldsymbol{\theta}}


% COMMENTING
\usepackage[usenames]{color}
\newcommand{\comment}[2]{\textcolor{blue}{\bf #1: #2}}
\newcommand{\problem}[2]{\textcolor{red}{\bf #1: #2}}
\newcommand{\todo}[2]{{\bf TO-DO: #1: #2}}

\def\firstplace{Institute 1, University 1, City 1, Country 1}
\def\secondplace{Institute 2, University 2, City 2, Country 2}
\def\thirdplace{Institute 3, University 3, City 3, Country 3}
\def\fourthplace{Institute 4, University 4, City 4, Country 4}
\def\fifthplace{Institute 5, University 5, City 5, Country 5}
\def\sixthplace{Institute 6, University 6, City 6, Country 6}

\def\email{\tt first.author@domain.name}


%%%%%%%%%%%%%%%%%%%%%%%%%%%%%%%%%%%%%%%%%%%%%%%%%%%%%%%%%%%%%%%%%%%%%%%%%%%%%%

\title[Short title]
{The long version of the title of this paper}
    
\author[All of us]{%
  First Author,$^{1}$\thanks{\email}
  Second Author,$^{2}$
  Third Author,$^{3}$
\newauthor{%
  Fourth Author,$^{4}$
  Fifth Author,$^{5}$
  Sixth Author.$^{6}$}
  \medskip\\
  $^1$\firstplace\\
  $^2$\secondplace\\
  $^3$\thirdplace\\
  $^4$\fourthplace\\
  $^5$\fifthplace\\
  $^6$\sixthplace\\
}

%%%%%%%%%%%%%%%%%%%%%%%%%%%%%%%%%%%%%%%%%%%%%%%%%%%%%%%%%%%%%%%%%%%%%%%%%%%%%%

\begin{document}
             
\date{To be submitted to MNRAS}
             
\pagerange{\pageref{firstpage}--\pageref{lastpage}}\pubyear{2XXX}

\maketitle

\label{firstpage}

%%%%%%%%%%%%%%%%%%%%%%%%%%%%%%%%%%%%%%%%%%%%%%%%%%%%%%%%%%%%%%%%%%%%%%%%%%%%%%

\begin{abstract}
Write this first.
\end{abstract}

\begin{keywords}
  Need keywords.
\end{keywords}

%%%%%%%%%%%%%%%%%%%%%%%%%%%%%%%%%%%%%%%%%%%%%%%%%%%%%%%%%%%%%%%%%%%%%%%%%%%%%%

\section{Introduction}

Goes here.

We are interested in the following questions:

\begin{itemize}

\item First question?

\item Second question? 

\item Third question? 

\end{itemize}

This paper is organized as follows. In \Sref{sec:first} we \ldots
In \Sref{sec:discuss} we present our conclusions.


%%%%%%%%%%%%%%%%%%%%%%%%%%%%%%%%%%%%%%%%%%%%%%%%%%%%%%%%%%%%%%%%%%%%%%%%%%%%%%

\section{Next section}
\label{sec:next}


%%%%%%%%%%%%%%%%%%%%%%%%%%%%%%%%%%%%%%%%%%%%%%%%%%%%%%%%%%%%%%%%%%%%%%%%%%%%%%

\section{Conclusions}
\label{sec:conclude}

Summarize briefly.

Our conclusions regarding \ldots can be stated as follows:

\begin{itemize}

\item First answer.

\item Second answer. 

\item Third answer. 

\end{itemize}

Wrap up.


%%%%%%%%%%%%%%%%%%%%%%%%%%%%%%%%%%%%%%%%%%%%%%%%%%%%%%%%%%%%%%%%%%%%%%%%%%%%%%

\section*{Acknowledgements}

We thank A, B and C for useful discussions.
%
XXX acknowledges support from XXX.
%
YYY acknowledges support from YYY.



%%%%%%%%%%%%%%%%%%%%%%%%%%%%%%%%%%%%%%%%%%%%%%%%%%%%%%%%%%%%%%%%%%%%%%%%%%%%%%
% MNRAS does not use bibtex, input .bbl file instead. Generate this in the
% makefile using bubble script in scriptutils:

%%%%%%%%%%%%%%%%%%%%%%%%%%%%%%%%%%%%%%%%%%%%%%%%%%%%%%%%%%%%%%%%%%%%%%%%%%%%%%
% 
% GREAT3 Weak Lensing Challenge Entry
% 
%%%%%%%%%%%%%%%%%%%%%%%%%%%%%%%%%%%%%%%%%%%%%%%%%%%%%%%%%%%%%%%%%%%%%%%%%%%%%%

\documentclass[useAMS,usenatbib]{mn2e}

\voffset=-0.8in

% Packages:
\usepackage{graphicx}
\usepackage{amsmath}
\usepackage{xspace}

% Macros:
\input{macros.tex}
\input{addresses.tex}

%%%%%%%%%%%%%%%%%%%%%%%%%%%%%%%%%%%%%%%%%%%%%%%%%%%%%%%%%%%%%%%%%%%%%%%%%%%%%%

\title[Short title]
{The long version of the title of this paper}
    
\author[All of us]{%
  First Author,$^{1}$\thanks{\email}
  Second Author,$^{2}$
  Third Author,$^{3}$
\newauthor{%
  Fourth Author,$^{4}$
  Fifth Author,$^{5}$
  Sixth Author.$^{6}$}
  \medskip\\
  $^1$\firstplace\\
  $^2$\secondplace\\
  $^3$\thirdplace\\
  $^4$\fourthplace\\
  $^5$\fifthplace\\
  $^6$\sixthplace\\
}

%%%%%%%%%%%%%%%%%%%%%%%%%%%%%%%%%%%%%%%%%%%%%%%%%%%%%%%%%%%%%%%%%%%%%%%%%%%%%%

\begin{document}
             
\date{To be submitted to MNRAS}
             
\pagerange{\pageref{firstpage}--\pageref{lastpage}}\pubyear{2XXX}

\maketitle

\label{firstpage}

%%%%%%%%%%%%%%%%%%%%%%%%%%%%%%%%%%%%%%%%%%%%%%%%%%%%%%%%%%%%%%%%%%%%%%%%%%%%%%

\begin{abstract}
Write this first.
\end{abstract}

\begin{keywords}
  Need keywords.
\end{keywords}

%%%%%%%%%%%%%%%%%%%%%%%%%%%%%%%%%%%%%%%%%%%%%%%%%%%%%%%%%%%%%%%%%%%%%%%%%%%%%%

\section{Introduction}

Goes here.

We are interested in the following questions:

\begin{itemize}

\item First question?

\item Second question? 

\item Third question? 

\end{itemize}

This paper is organized as follows. In \Sref{sec:first} we \ldots
In \Sref{sec:discuss} we present our conclusions.


%%%%%%%%%%%%%%%%%%%%%%%%%%%%%%%%%%%%%%%%%%%%%%%%%%%%%%%%%%%%%%%%%%%%%%%%%%%%%%

\section{Next section}
\label{sec:next}


%%%%%%%%%%%%%%%%%%%%%%%%%%%%%%%%%%%%%%%%%%%%%%%%%%%%%%%%%%%%%%%%%%%%%%%%%%%%%%

\section{Conclusions}
\label{sec:conclude}

Summarize briefly.

Our conclusions regarding \ldots can be stated as follows:

\begin{itemize}

\item First answer.

\item Second answer. 

\item Third answer. 

\end{itemize}

Wrap up.


%%%%%%%%%%%%%%%%%%%%%%%%%%%%%%%%%%%%%%%%%%%%%%%%%%%%%%%%%%%%%%%%%%%%%%%%%%%%%%

\section*{Acknowledgements}

\input{acknowledgments.tex}

%%%%%%%%%%%%%%%%%%%%%%%%%%%%%%%%%%%%%%%%%%%%%%%%%%%%%%%%%%%%%%%%%%%%%%%%%%%%%%
% MNRAS does not use bibtex, input .bbl file instead. Generate this in the
% makefile using bubble script in scriptutils:

\input{ms.bbl}

%%%%%%%%%%%%%%%%%%%%%%%%%%%%%%%%%%%%%%%%%%%%%%%%%%%%%%%%%%%%%%%%%%%%%%%%%%%%%%

\label{lastpage}
\bsp

\end{document}

%%%%%%%%%%%%%%%%%%%%%%%%%%%%%%%%%%%%%%%%%%%%%%%%%%%%%%%%%%%%%%%%%%%%%%%%%%%%%%


%%%%%%%%%%%%%%%%%%%%%%%%%%%%%%%%%%%%%%%%%%%%%%%%%%%%%%%%%%%%%%%%%%%%%%%%%%%%%%

\label{lastpage}
\bsp

\end{document}

%%%%%%%%%%%%%%%%%%%%%%%%%%%%%%%%%%%%%%%%%%%%%%%%%%%%%%%%%%%%%%%%%%%%%%%%%%%%%%


%%%%%%%%%%%%%%%%%%%%%%%%%%%%%%%%%%%%%%%%%%%%%%%%%%%%%%%%%%%%%%%%%%%%%%%%%%%%%%

\label{lastpage}
\bsp

\end{document}

%%%%%%%%%%%%%%%%%%%%%%%%%%%%%%%%%%%%%%%%%%%%%%%%%%%%%%%%%%%%%%%%%%%%%%%%%%%%%%


%%%%%%%%%%%%%%%%%%%%%%%%%%%%%%%%%%%%%%%%%%%%%%%%%%%%%%%%%%%%%%%%%%%%%%%%%%%%%%

\label{lastpage}
\bsp

\end{document}

%%%%%%%%%%%%%%%%%%%%%%%%%%%%%%%%%%%%%%%%%%%%%%%%%%%%%%%%%%%%%%%%%%%%%%%%%%%%%%
