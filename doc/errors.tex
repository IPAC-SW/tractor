\documentclass[letterpaper]{article}
\usepackage{amssymb, amsmath}
\usepackage{hyperref}

\newcommand{\likelihood}{\mathcal{L}}
\newcommand{\dd}{\partial}
\newcommand{\var}{\textrm{var}}

\newcommand{\niceurl}[1]{\mbox{\href{#1}{\textsl{#1}}}}

\begin{document}

\title{Uncertainty estimates in \emph{The Tractor}}
\author{Dustin Lang}
\date{\today}

\maketitle

\section{Uncertainty estimates}

The uncertainty estimates produced by \emph{The Tractor} are related
to the Cram\'er--Rao bound.  Specifically, the Cram\'er--Rao bound is
a lower bound on the variance of any unbiased estimator $\hat{\theta}$:
\begin{equation}
\var(\hat{\theta}) \ge \displaystyle\frac{1}{I(\theta)}
\end{equation}
where $I(\theta)$ is the Fisher information,
\begin{align}
  I(\theta) &= E_{x}\left[ \left( \frac{\dd \log \likelihood(x | \theta)}{\dd \theta} \right)^2 \right] \\
  & = - E_{x}\left[ \frac{\dd^2 \log \likelihood(x | \theta)}{\dd \theta^2} \right]
\end{align}
where $\likelihood(x | \theta)$ is the likelihood of the data $x$ given
the parameter we are estimating, $\theta$,
as outlined in the Wikipedia articles.\footnote{%
  \niceurl{https://en.wikipedia.org/wiki/Cram\%C3\%A9r\%E2\%80\%93Rao\_bound}
  and
  \niceurl{https://en.wikipedia.org/wiki/Fisher\_information}}


In the case of \emph{The Tractor}, the likelihood is based on a simple
independent Gaussian noise model
\begin{equation}
\likelihood(x | \theta) = \exp\left(-\sum_i \frac{(m(\theta)_i - x_i)^2}{2 \sigma_i^2} \right)
\end{equation}
where $m(\theta)$ is our pixel-space model prediction of the
appearance of an astronomical object in an image, given model
parameters $\theta$, and the observed pixel values are $x_i$.  The sum
is over pixels $i$, possibly in several overlapping images, and
$\sigma_i$ are the per-pixel noise values.

The log likelihood is trivially
\begin{equation}
\log \likelihood(x | \theta) = -\sum_i \frac{(m(\theta)_i - x_i)^2}{2 \sigma_i^2}
\end{equation}
with first derivative
\begin{equation}
\frac{\dd}{\dd \theta} \log \likelihood(x | \theta) = - \sum_i \frac{m(\theta)_i - x_i}{\sigma_i} \frac{1}{\sigma_i} \frac{\dd}{\dd \theta} m(\theta)_i
\quad ,
\end{equation}
which should equal zero when we have correctly maximized the likelihood.
The second derivative is
\begin{equation}
\frac{\dd^2}{\dd \theta^2} \log \likelihood(x | \theta) = - \sum_i \left[
  \frac{m(\theta)_i - x_i}{\sigma_i} \frac{1}{\sigma_i} \frac{\dd^2}{\dd \theta^2} m(\theta)_i
  + \left( \frac{1}{\sigma_i} \frac{\dd}{\dd \theta} m(\theta)_i \right)^2
\right]
\quad ,
\end{equation}
where the first term is zero.

Returning to the Cram\'er--Rao bound, we have
\begin{align}
\frac{1}{\var(\hat{\theta})} \le 
& - E_{x}\left[ \frac{\dd^2 \log \likelihood(x | \theta)}{\dd \theta^2} \right]
\end{align}
and since our second derivative is independent of $x$, the expectation
collapses and we get
\begin{align}
\frac{1}{\var(\hat{\theta})} & \le 
  \sum_i \left( \frac{1}{\sigma_i} \frac{\dd}{\dd \theta} m(\theta)_i \right)^2
  \quad ,
\end{align}
which is the inverse-variance estimate we use in \emph{The Tractor}.

Now, the derivative $\frac{\dd}{\dd \theta} m(\theta)_i$ is just the
derivative of the pixel-space model prediction with respect to the
parameter; we have to compute this (many times at different $\theta$)
during the optimization.

The Cram\'er--Rao bound is the smallest variance an unbiased estimator
can achieve.  Since in \emph{The Tractor} we are working in likelihood
space, we can actually expect to achieve the bound.





% According to Wikipedia, refs are
%
% Cramér, Harald (1946). Mathematical Methods of Statistics. Princeton, NJ: Princeton Univ. Press. ISBN 0-691-08004-6. OCLC 185436716.
%
% Rao, Calyampudi Radakrishna (1945). "Information and the accuracy attainable in the estimation of statistical parameters". Bulletin of the Calcutta Mathematical Society 37: 81–89. MR 0015748.

\end{document}
